\chapter{Einleitung}
\section{FRAP}
\paragraph{Was ist FRAP und was mache ich damit?}
\paragraph{Wie funktioniert FRAP?}
\paragraph{Wie werden FRAP-Kurven üblicherweise analysiert?}
und was mache ich anders :P\\
Bspw. fitten an Formel zur Berechnung von Dauer der recovery
\newline
\newline

Mit zunehmender Miniaturisierung spielen stochastische Vorgänge immer größere Rolle

$\rightarrow$ stochastisches Modell konstruiert und simuliert werden und beispielhaft an reale FRAP-Kurven gefittet werden, um einen Eindruck zu kriegen ob es sinnvoll ist stochastisch zu simulieren. Wie aussagekräftig sind die Daten?
\newline
\par
Die Fluoreszenzwiederherstellung nach erfolgtem Photobleichen (engl.: Fluorescence Recovery After Photobleaching (FRAP)) ist
eine Methode zur Bestimmung der Diffusions- und Wechselwirkungseigenschaften von bestimmten Medien in der Biologie und
der Materialwissenschaft im Mikrometerbereich.
Die Methode wurde in den 1970er Jahren entwickelt und erlebte in den 1990er Jahren einen .
FRAP basiert auf einem einfachen Ansatz:
Zunächst wird der Translationsdiffusionskoeffizient eines fluoreszenzmarkierten Proteins durch Bleichen von Molekülen gemessen,
die in ein begrenztes Volumen eines Lichtstrahls diffundieren [20]. Infolgedessen nimmt die Fluoreszenzintensität im gebleichten
Bereich ab, wie in Fig. 1 (b) dargestellt.\par
Dieser Vorgang wird als Photobleichen bezeichnet [19]. Im nächsten Schritt kann die Fluoreszenzwiederherstellung mit einem stark
abgeschwächten Lichtstrahl aufgrund der Diffusion fluoreszenzmarkierter Moleküle aus den benachbarten ungebleichten Bereichen in
den gebleichten Bereich gemessen werden [20]. Die Ergebnisse der über die Zeit gemessenen unterschiedlichen Intensitäten der
Fluoreszenz werden durch eine Erholungskurve dargestellt (Abb. 1 (i)) [19]. Die Wiederfindungsrate der Fluoreszenzintensität
hängt von der Beweglichkeit der Moleküle ab. Das heißt, wenn alle Moleküle im Wesentlichen beweglich sind, erreicht die endgültige
Fluoreszenzintensität - nach einer bestimmten Zeit, die sich aus dem Bleichprozess ergibt - nahezu das gleiche Intensitätsniveau
wie vor dem Bleichen, siehe 1 (c) (d) ( ich). Eine Divergenz zwischen der endgültigen Fluoreszenzintensität und der vor dem
Bleichen zeigt an, dass ein Anteil an unbeweglichen gebleichten Partikeln vorhanden ist (1 (i)). Mit einer geeigneten
mathematischen Methode kann man die Entwicklung der Fluoreszenzwiederherstellung analysieren und quantitative Informationen über
die Molekulardynamik wie Diffusion und Bindung extrahieren [19]. Nächster Zur Berechnung des Diffusionskoeffizienten D, der ein
Maß für die Mobilität von Atomen ist und ein hohes Maß an Kenntnis der FRAP-Theorie erfordert, können FRAP-Messungen anhand der
definierten Halbwertszeit der Erholung t 1/2 [14] quantifiziert werden als die Zeit, um die Hälfte der endgültig
wiederhergestellten Fluoreszenz zu erreichen. Es kann leicht extrahiert werden FRAP-Erholungskurven [10,14]. Der einfachste Weg,
eine Erholungskurve zu beschreiben, ist beispielsweise die Verwendung einer einfachen Expontentialfunktion [15, 19].


\section{Promyelotic Leukemia Nuclear Bodies}
Promyelotic Leukemia nuclear bodies (oder kurz PML NBs) sind Proteinkomponenten, die als wesentlich bei Leukämie festgestellt wurden und bei der Translation eine wichtige Rolle spielen. Sie weiten das Chromatin...
\begin{itemize}
	\item sind Proteinkomplexe im Zellkern
	\item bestehen aus verschiedenen Untereinheiten PML I bis PML VI
	\item sind wichtig, um Raum für die Translation zu schaffen
	\item haben eine hohe Mobilität
	\item werden in der Regel innerhalb von Minuten auf- und wieder abgebaut
\end{itemize}