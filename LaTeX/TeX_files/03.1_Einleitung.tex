\chapter{Einleitung}
Biosensoren sind aus der heutigen Welt nicht mehr wegzudenken. Besonders in der Medizintechnik sind sie als Bestandteile von Analysegeräten zum Beispiel für Blut oder den Darminhalt unverzichtbar geworden. Aber auch in der weiterführenden Forschung zu Genom, Transkriptom oder Proteom sind Biosensoren ein sehr wichtiger Bestandteil, ohne den eine Analyse in dem heute üblichen Umfang nicht möglich wäre.
Dabei ist die Definition und Benennung eines Biosensors keinesfalls eindeutig geklärt. Es gibt verschiedene Ansätze die existierenden Biosensoren zu klassifizieren.

\section{Biosensor}
Nach Definition der International Union of Pure and Applied Chemistry (IUPAC) ist ein Biosensor ein in sich geschlossenes Gerät, das spezifische quantitative oder semi-quantitative analytische Informationen bereitstellen kann. Dies geschieht unter Verwendung eines biologischen Erkennungselements (auch: biochemischer Rezeptor), welches in direktem räumlichem Kontakt mit einem Transduktionselement steht. Biosensoren sind zudem von bioanalytischen Systemen zu unterscheiden, da sie ohne Zugabe von zusätzlichen Reagenzien wiederholt kalibriert werden können. Ein Biosensor, der nach einmaligem Gebrauch wegwerfbar, oder nicht in der Lage ist die Analytkonzentration nach kurzer Regeneration kontinuierlich zu überwachen, sollte als Einweg-Biosensor bezeichnet werden.\par
Biosensoren können nach der Art ihres Rezeptors oder nach der Art ihrer physikalisch-chemischen Signalübertragung klassifiziert werden. Das biologische Erkennungselement kann auf einer chemischen Reaktion basieren, die durch Makromoleküle katalysiert wird, oder auf einer Gleichgewichtsreaktion mit diesen beruht. Die Makromoleküle können wiederum isoliert oder synthetisiert worden sein, oder natürlich vorkommen. Im Gleichgewicht gibt es keinen Nettoverbrauch des oder der Analyten durch die im Sensor verankerten Biokomplexe.\par
Biosensoren können weiter nach den Analyten oder Reaktionen die sie überwachen sollen klassifiziert werden. Dabei können sie die Analytkonzentration direkt überwachen oder die Reaktionen, die die entsprechenden Analyten produzieren oder verbrauchen. Alternativ kann eine indirekte Überwachung durch die Inhibition oder Aktivierung des biochemischen Rezeptors erreicht werden.\cite{definitionbiosensors}
%\begin{quotation}
%	\textit{A biosensor is a self-contained integrated device, which is capable of providing specific quantitative or semi-quantitative analytical information using a biological recognition element (biochemical receptor), which is retained in direct spatial contact with a transduction element. Because of their ability to be repeatedly calibrated, we recommend that a biosensor should be clearly distinguished from a bioanalytical system, which requires additional processing steps, such as reagent addition. A device that is both disposable after one measurement, i.e., single use, and unable to monitor the analyte concentration continuously or after rapid and reproducible regeneration should be designated as a single-use biosensor.\\
%	Biosensors may be classified according to their biological specificityconferring mechanism or, alternatively, to their mode of physico-chemical signal transduction. The biological recognition element may be based on a chemical reaction catalyzed by, or on an equilibrium reaction with, macromolecules that have been isolated, engineered or are present in their original biological environment. In the latter cases, equilibrium is generally reached and there is no further, if any, net consumption of the analyte(s) by the immobilized biocomplexing agent incorporated into the sensor. Biosensors may be further classified according to the analytes or reactions that they monitor by directly monitoring the analyte concentration or by reactions producing or consuming such analytes; alternatively, an indirect monitoring of an inhibitor or the activator of the biological recognition element (biochemical receptor) may be achieved.}
%\end{quotation}

\section{Aufbau Biosensor}
Ein Biosensor besteht im Allgemeinen aus drei Teilen. Einer Kammer in die das zu untersuchende Gemisch eingebracht wird, einer biologischen Schicht (engl.: biological recognition layer), zur speziellen Analyse der Bestandteile dieses Gemisches, die dann fest auf einem physikochemischen Wandler (engl.: transducer) aufgebracht ist.\cite{AfullyintegratedCMOSfluorescencebiochip} Die biologische Erkennungsschicht kann speziell angefertigte Sonden (engl.: probes) enthalten, um DNA oder andere Biomoleküle aus dem darüber liegenden Volumen zu binden und dann ein für den transducer messbares Signal zu erzeugen.
Allgemein kann man den Aufbau eines Biosensors wie folgt illustrieren:
\begin{figure}[h]
	\includegraphics[height=45mm, width=150mm]{Bilder/Biosensor}
	\caption[justification=raggedright]{Schematische Darstellung eines Biosensors}
\end{figure}

\section{Arten}
Wie in der Definition (siehe oben) erwähnt, können Biosensoren auf unterschiedliche Art und Weise klassifiziert werden. %interner Verweis?
Dies kann nach Art des transducers, nach der Beschaffenheit des recognition layers oder ganz allgemein nach der Messtechnik geschehen. Affinitätsbiosensoren oder potentiometrische Biosensoren sind beispielsweise solche Begriffe, die aussagen welches Verfahren im Biosensor zur Analyse angewendet wird, jedoch noch keine Informationen darüber enthalten, welcher transducer oder welche biologische Komponente verwendet wurden. Am transducer selbst kann man zusätzlich beispielsweise nach der verwendeten Materialart unterscheiden. Am recognition layer kann bezüglich der verwendeten biologischen Komponente(n) (z.B. DNA-primer, Enzyme) oder den zu detektierenden Analyten (z.B. Viren, Metaboliten, Teile von DNA) unterschieden werden.\cite{chemicalsensorsandbiosensors}
Einige Biosensoren, die viele Anwendungen finden, sind Affinitätsbiosensoren, optische Biosensoren und potentiometrische Biosensoren.
Affinitätsbiosensoren funktionieren nach dem Schlüssel-Schloss-Prinzip, wodurch sie in der Lage sind ganz gezielt eine bestimmte Art von Analyten aus einer Molekülsuspension zu binden und über den transducer detektieren zu lassen.\par
Optische Biosensoren detektieren, wie der Name es schon sagt, das Licht. Dieses wird von den Analyten selbst ausgestrahlt, auf dem Weg von einer externen Lichtquelle zum transducer entweder von den vorhandenen Analyten reflektiert oder absorbiert. Wird das Licht nach Absorption später mit geringerer Energie wieder abgegeben, spricht man von Lumineszenz.
In dieser Arbeit wird gezielt auf optoelektronische Affinitätsbiosensoren eingegangen.

\section{Anwendung von Biosensoren}
Biosensoren werden in der Medizin- und Umwelttechnik oftmals in großer Zahl parallel auf Biochips verwendet. % Definition Biochip
Ein Biochip ist ein durch CMOS\footnote{CMOS: complementary metal-oxide semiconductor}-Technik hergestelltes Gerät zur biochemischen Analyse, das aus einem Pixelarray und einem Mikroprozessor, zur direkten Signalverarbeitung aus den einzelnen Pixeln, besteht. Ein Pixel stellt dabei eine biosensorisch aktive Einheit (Biosensor) dar.
Durch die parallele Verwendung der einzelnen Sensoren sind einzelne Messreihen unter annähernd gleichen Bedingungen durchführbar. Die Anzahl der Einzelversuche kann dadurch, im Rahmen der technischen Möglichkeiten, beliebig skaliert werden. Zusätzlich sinken die durchschnittlichen Kosten für einen einzelnen Biosensor.\cite{CMOSBio}
Ein Problem, dass Biochips durch jeden einzelnen Biosensor haben, ist das Rauschen der einzelnen Bauteile. Das heißt, dass das Signal, welches am Ende die Messvorrichtung verlässt, bei konstant bleibenden Bedingungen in einem gewissen Maß schwankt. Dies kann durch Elektronenkollisionen in den elektrischen Leitungen verursacht werden, aber, speziell bei Biosensoren, auch durch das biologische System.
Um dieses Rauschen vom eigentlichen Signal, welches gemessen werden soll, zu unterscheiden muss man das Rauschen also charakterisieren und versuchen es durch Vorüberlegungen möglichst gering zu halten. Das technische Rauschen\footnote{erstmals beschrieben von Walter Schottky, 1918\cite{Schottky1918}} ist dem biologischen Rauschen in dieser Hinsicht ein paar Jahrzehnte voraus.
