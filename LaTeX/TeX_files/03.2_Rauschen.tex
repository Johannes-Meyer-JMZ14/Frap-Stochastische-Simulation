\chapter{Rauschen an optoelektronischen Biosensoren}

\section{Rauschen}
Das Signalrauschen ist für jeden Sensor ein Problem und entsteht durch physikalisch oder chemisch bedingte Übermittlungsinterferenzen des zu messenden Signals über verschiedene Bauteile bis zur finalen, menschenlesbaren Form, beispielsweise am Computer. Das Gesamtrauschen setzt sich dabei aus den Rauschkomponenten der einzelnen Bauteile zusammen.\newline
Speziell bei den Biosensoren kommt zu den technischen Rauschquellen noch eine rauschende biologische Komponente hinzu.\par

Technisches Rauschen sind beispielsweise das weiße (thermische) und rosa Rauschen (Kollisionsrauschen der Leitungselektronen oder auch 1/f-Rauschen)\footnote{f ist die Frequenz (Anzahl der Wiederholungen eines periodischen Vorgangs pro Zeiteinheit, meist in Hertz: $1Hz = \frac{1}{s}$)}. Weißes Rauschen hat bei gleich bleibender Frequenz ein konstantes Leistungsspektrum (in dB angegeben) und ist proportional zur absoluten Temperatur, sowie unabhängig vom Leitungsmedium. Das rosa Rauschen nimmt mit steigender Frequenz im Leistungsspektrum mit ungefähr 1/f ab (daher auch der Name) und dominiert daher in niedrigen Frequenzen gegenüber dem thermischen Rauschen.\par

Bei technischen Bauteilen ist relativ einfach unter verschiedenen Bedingungen, wie anliegender Spannung, Stromfluss oder Temperatur, eine Leermessung zu machen und so festzustellen, wie stark das jeweilige Bauteil rauscht.

\section{Biologisches Rauschen}
Bei biologischen Systemen (im Biosensor) ist es nicht möglich eine sogenannte Leermessung durchzuführen. Dadurch ist es schwierig die Rauschleistung von biologischen Systemen vorherzusagen, auch weil sich das System bei kleinen Änderungen in der Umwelt komplett anders verhalten kann, beispielsweise durch Überschreitung eines bestimmten Schwellwertes. Deshalb ist es unter anderem für die Sensorentwicklung sehr wichtig vorab die Rauschleistung des biologischen Systems ermessen zu können, um die Kalibrierungen der einzelnen Bauteile entsprechend vorzunehmen.\par

Biologisches Rauschen wird in verschiedenen Anwendungen verschieden beschrieben. Aber egal ob in der mirkoskopischen (DNA-Analyse) oder mesoskopischen (Genexpression) Anwendung, allen gemein ist die über Zeit stattfindende Änderung des jeweiligen biologischen Systems. Dass es diese Änderung gibt ist ganz natürlich, da kein biologisches System wirklich still steht, sondern maximal um einen Gleichgewichtszustand (engl.: steady state) schwankt.\newline
Bei Biosensoren entspricht das biologische Rauschen (engl.: biological shot noise) der messbaren Änderung des Signals durch die biologische Komponente. Aus den Arbeiten von A. Hassibi geht hervor, dass es durch die Teilchenbewegung, Diffusion, Kollisionen freier Moleküle und Bindungsereignisse zu solchem Rauschen kommen kann.\cite{biological_shot-noise_and_SNR_in_affinity-based_biosensors} Die einzelnen möglichen biologischen Rauschquellen sind je nach Art des Biosensors mehr oder weniger ausschlaggebend für das Gesamtrauschen.\par

Bei optoelektronischen Biosensoren werden die Analyten aus dem Reaktionsvolumen mit Hilfe von Licht detektiert. Um ein möglichst klares Signal zu erhalten werden dazu die Analyten am recognition layer immobilisert und dann vermessen. Dazu wird entweder eine Absorbtionsmessung, bei der anhand des nicht durchdringenden Lichts die Anzahl der Analyten bestimmt wird, oder eine Fluoreszenzmessung\cite{AfullyintegratedCMOSfluorescencebiochip}, bei der nach der Belichtung das auf niedrigerer Wellenlänge von den, unter Umständen mit entsprechenden Fluoreszenzmarkern versehenen, Analyten emittierte Licht aufgefangen wird, durchgeführt.\par

Da es bei den optoelektronischen Biosensoren einen wesentlichen Unterschied macht, ob die Analyten direkt am recognition layer sitzen, oder nicht, sind die Bindungsereignisse ein wesentlicher Teil des biologischen Raschens in diesen Sensoren. Daher beschäftigt sich diese Arbeit mit den Bindungsereignissen an der biologischen Erkennungsschicht, welche sich mit dem Gillespie-Algorithmus besonders realistisch smiulieren lassen.\par
