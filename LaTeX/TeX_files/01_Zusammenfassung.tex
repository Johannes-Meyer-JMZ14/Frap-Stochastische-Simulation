\addchap{Zusammenfassung}
\thispagestyle{empty}
Das Signal von optoelektronischen Biosensoren wird von biologischem Rauschen beeinflusst. Dieses kann je nach biologischem System und äußeren Bedingungen unterschiedlich stark ausgeprägt sein.\par

In der Sensortechnik ist ein wesentlicher Teil der Entwicklung eines Sensors die Bestimung der Gesamtrauschleistung des jeweiligen Messsystems, um die Bauteile zur Signalanalyse bestmöglich aufeinander und das Signal abstimmen zu können. In dieser Arbeit wird mit Hilfe einer Implementierung des Gillespie--Algorithmus eine Simulation zum biologischen Rauschen umgesetzt. In dieser werden gezielt Bindungsreaktionen mit einer Erkennungsschicht, genauer mit den in ihr enthaltenen \emph{Sonden}, simuliert.\par

Weiterhin werden die Auswirkungen verschiedener Eingabeparameter des Gillespie--Algorithmus auf das Gesamtrauschverhalten mittels Monte-Carlo-Simulationen in Diagrammen abgebildet.\par

%Schlussendlich werden die Gillespie--Simulationen mit den Daten von Arjang Hassibi verglichen, der auf dem Gebiet des biologischen Rauschens einiges an Arbeit geleistet hat.