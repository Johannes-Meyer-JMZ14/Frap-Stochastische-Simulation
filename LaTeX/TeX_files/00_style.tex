\documentclass[a4paper,12pt,twoside]{scrbook}
\usepackage[ngerman]{babel} % deutsches Sprachpaket mit neuer deutscher Rechtschreibung
\usepackage[utf8]{inputenc} % Umlaute
\usepackage[T1]{fontenc} % Erweiterter Zeichensatz

\usepackage{geometry} % Seitenränder
\geometry{left=4cm, right=2cm, top=3cm, bottom=3cm}
% Achtung! Beim Drucker muss Seitenbereich anpassen ausgestellt sein.

\usepackage{listings}
\lstset{numbers=left, numberstyle=\tiny, numbersep=5pt, language=Python}

\usepackage{soulutf8} % für gesperrte Textabschnitte
\usepackage{setspace} % Einstellen des Zeilenabstandes
\parindent0pt % \setlength{\parindent}{0mm} % kein Einrücken der ersten Zeile eines Absatzes
\parskip1ex % \setlength{\parskip}{1mm} % Abstand zwischen den Absätzen

\usepackage{cite}
\usepackage[german=quotes]{csquotes} % deutsche Anführungszeichen
% \usepackage{scrextend}
\usepackage{graphicx}
\usepackage[]{caption}
\usepackage{amsmath}
\usepackage{amssymb}
\usepackage[hidelinks]{hyperref} % als letztes laden, macht alle Links im PDF anklickbar
\hypersetup{pdftitle={Stochastische Simulation zur Vorhersage von biologischem Rauschen auf optoelektronischen Biosensoren}, pdfauthor={Johannes Meyer}, pdfsubject={biologisches Rauschen}, pdfkeywords={biologisches Rauschen, Biosensor, Anwendung Gillespie}}